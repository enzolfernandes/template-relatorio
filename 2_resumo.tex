\newpage

\section{Resumo}

A experiência consiste no estudo de um circuito RLC paralelo cujo comporta-se como passa-faixa e
como filtro rejeita-faixa, dependendo da tensão de saída análisada (indutor e resistor, respectivamente). Para isso, o sistema foi simulado em um software adequado, e foram analisadas as funções transferência , seu módulo, as frequências de corte, e ressonância. Estes filtros são circuitos capazes de selecionar ou excluir um sinal de sua saída
através dos sinais de entrada. A análise é dada através de gráficos gerados no software, de onde é
possível obter valores e compará-los com resultados de cálculos teóricos. O valor da frequências de corte e de ressonância obtidas
experimentalmente é comparada com o valor obtido através de cálculos, de maneira que o maior erro entre esses seja de
$0,462\%$, o que garante a veracidade dos dados e processos realizados nesta experiência.

\pagebreak